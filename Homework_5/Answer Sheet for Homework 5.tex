\documentclass[12pt]{article}

% Use packages %
\usepackage{graphicx, courier, amsmath, amssymb, amscd, amsfonts, mathtools, bm, esint, leftidx, extarrows, latexsym, relsize, color, tikz, comment, stmaryrd}
\usepackage[obeyspaces]{url}% http://ctan.org/pkg/url

% Set length %
\setlength{\textwidth}{160mm}
\setlength{\textheight}{235mm}
\setlength{\oddsidemargin}{-0mm}
\setlength{\topmargin}{-10mm}

% Define h-bar %
\newsavebox{\myhbar}
\savebox{\myhbar}{$\hbar$}
\renewcommand*{\hbar}{\mathalpha{\usebox{\myhbar}}}

% Chinese input %
%\usepackage{xeCJK} 
%\setCJKmainfont{微軟正黑體}
%\usepackage[T1]{fontenc}
%\makeatletter

% Equation number %
%\@addtoreset{equation}{section} 
%\renewcommand\theequation{{\thesection}.{\arabic{equation}}}
%\makeatletter 

% Helper Command %
\newcommand{\argmin}{\operatornamewithlimits{argmin}}
\newcommand{\rmnum}[1]{\romannumeral #1} 
\newcommand{\Rmnum}[1]{\expandafter\@slowromancap\romannumeral #1@}
\newcommand{\overbar}[1]{\mkern 1.5mu\overline{\mkern-1.5mu#1\mkern-1.5mu}\mkern 1.5mu}
\makeatother
\newcommand*{\QEDA}{\hfill\ensuremath{\blacksquare}}
\newcommand*{\QEDB}{\hfill\ensuremath{\square}}
\newcommand*{\BmVert}{\bigm\vert}
\newcommand{\bigslant}[2]{{\raisebox{.2em}{$#1$}\left/\raisebox{-.2em}{$#2$}\right.}}
\newcommand{\Nelements}[3]{\left\{ #1, ~ #2, \ldots, ~ #3 \right\}}
\newcommand{\CBrackets}[1]{\left\{#1\right\}}
\newcommand{\SBrackets}[1]{\left[#1\right]}
\newcommand{\BooBrackets}[1]{\left\llbracket#1\right\rrbracket}
\newcommand{\ParTh}[1]{\left(#1\right)}
\newcommand{\Ceil}[1]{\left\lceil#1\right\rceil}
\newcommand{\Floor}[1]{\left\lfloor#1\right\rfloor}
\newcommand{\BF}[1]{{\bf#1}}
\newcommand{\Inverse}[1]{{#1}^{-1}}
\newcommand{\Generator}[1]{\left\langle#1\right\rangle}
\newcommand{\AbsVal}[1]{\left|#1\right|}
\newcommand{\VecAbsVal}[1]{\left\|#1\right\|}
\newcommand{\BSlash}[2]{\left.#1\middle\backslash#2\right.}
\newcommand{\Divide}[2]{\left.#1\middle/#2\right.}
\newcommand{\SciNum}[2]{#1\times{10}^{#2}}
\newcommand{\Matrix}[2]{\SBrackets{\begin{array}{#1}#2\end{array}}}
\newcommand{\MatrixTwo}[4]{\ParTh{\begin{array}{cc}{#1}&{#2}\\{#3}&{#4}\end{array}}}
\newcommand{\MatrixNByN}[1]{\Matrix{cccc}{{#1}_{11} & {#1}_{12} & \cdots & {#1}_{1n} \\ {#1}_{21} & {#1}_{22} & \cdots & {#1}_{2n} \\ \vdots & \vdots & \ddots & \vdots \\ {#1}_{n1} & {#1}_{n2} & \cdots & {#1}_{nn}}}
\newcommand{\ndiv}{\hspace{-4pt}\not|\hspace{2pt}}
\newcommand{\eqdef}{\xlongequal{\text{def}}}%
\newcount\arrowcount
\newcommand\arrows[1]{\global\arrowcount#1 \ifnum\arrowcount>0
\begin{matrix}\expandafter\nextarrow\fi}
\newcommand\nextarrow[1]{\global\advance\arrowcount-1 \ifx\relax#1\relax\else \xrightarrow{#1}\fi\ifnum\arrowcount=0 \end{matrix}\else\\\expandafter\nextarrow\fi}
\newcommand{\horrule}[1]{\rule{\linewidth}{#1}}

% Tikz settings %
\usetikzlibrary{shapes,arrows}
\tikzstyle{decision} = [diamond, draw, fill=white!20, text width=4.5em, text badly centered, node distance=3cm, inner sep=0pt]
\tikzstyle{block}    = [rectangle, draw, fill=white!20, text width=8em, text centered, rounded corners, minimum height=4em]
\tikzstyle{point}    = [fill = white!20, minimum size=0.5cm]
\tikzstyle{line}     = [draw, -latex']
\tikzstyle{mapsto}   = [draw, |->]
\tikzstyle{cloud}    = [draw, ellipse,fill=red!20, node distance=3cm, minimum height=2em]

\begin{document}

\baselineskip 6.5mm
\setlength{\parindent}{0pt}
\title{ 
\normalfont \normalsize 
\horrule{0.5pt} \\[0.4cm]
\huge { \Huge Machine Learning \\ \large Answer Sheet for Homework 5 }\\ % The assignment title
\horrule{2pt} \\ [0.5cm]
}
\author{ { \Large Da-Min HUANG } \\
{\small R04942045} \\
{\small\textit{Graduate Institute of Communication Engineering, National Taiwan University}}
}
%\date{}
%\allowdisplaybreaks[4]
\maketitle

\subsection*{Problem 1}

The hard-margin support vector machine is with $d+1$ variables. For soft-margin support vector machine, there are $N$ more variables $\xi_n$, $1\leq n\leq N$.

So soft-margin support vector machine is a quadratic programming problem with $N+d+1$ variables.

\QEDB

\horrule{0.5pt}

\subsection*{Problem 2}

I wrote a \url{Q02.py} to help me get the answer. By using Python package \url{cvxopt}$^{[2]}$, with
\begin{align}
\BF{z}=\Matrix{rr}{1&-2\\4&-5\\4&-1\\5&-2\\7&-7\\7&1\\7&1}, \BF{y}=\Matrix{c}{-1\\-1\\-1\\+1\\+1\\+1\\+1}
\end{align}
and
\begin{align}
\BF{Q}&=\Matrix{ccc}{0&0&0\\0&1&0\\0&0&1},\BF{p}=\Matrix{c}{0\\0\\0},\\
\BF{A}^T&=\Matrix{rrr}{-1&-1&2\\-1&-4&5\\-1&-4&1\\1&5&-2\\1&7&-7\\1&7&1\\1&7&1},\BF{c}=\Matrix{c}{1\\1\\1\\1\\1\\1\\1}
\end{align}
To use this package, I gave \url{solvers.qp(}$\BF{Q}, \BF{p}, -\BF{A}^T, -\BF{c}$\url{)} and got
\begin{align}
b=-9,\BF{w}=\SBrackets{2,0}
\end{align}
So the hyperplane is
\begin{align}
2z_1-9=0\Rightarrow z_1=4.5
\end{align}

\QEDB

\horrule{0.5pt}

\subsection*{Problem 3}

I wrote a \url{Q03.py} to help me get the answer. By using Python package \url{cvxopt}, with
\begin{align}
\BF{Q}&=\Matrix{rrrrrrr}{4&1&1&0&-1&-1&-1\\1&4&0&-1&-9&-1&-1\\1&0&4&-1&-1&-9&-1\\0&-1&-1&4&1&1&9\\-1&-9&-1&1&25&9&1\\-1&-1&-9&1&9&25&1\\-1&-1&-1&9&1&1&25},\BF{p}=\Matrix{r}{-1\\-1\\-1\\-1\\-1\\-1\\-1},\\
-\BF{A}^T&=\Matrix{rrrrrrr}{-1&0&0&0&0&0&0\\0&-1&0&0&0&0&0\\0&0&-1&0&0&0&0\\0&0&0&-1&0&0&0\\0&0&0&0&-1&0&0\\0&0&0&0&0&-1&0\\0&0&0&0&0&0&-1},\BF{c}=\Matrix{c}{0\\0\\0\\0\\0\\0\\0}
\end{align}
with
\begin{align}
\BF{G}=\BF{y}^T=\Matrix{ccccccc}{-1&-1&-1&1&1&1&1}\text{ and }h=0
\end{align}
and
To use this package, I gave \url{solvers.qp(}$\BF{Q}, \BF{p}, -\BF{A}^T, \BF{c},\BF{G},h$\url{)} and got
\begin{align}
\alpha=\SBrackets{\SciNum{4.32}{-9}\approx0,0.704,0.704,0.889,0.259,0.259,\SciNum{5.27}{-10}\approx0}
\end{align}
where \url{cvxopt} needs conditions
\begin{align}
-\BF{A}^T{\bm\alpha}\preceq\BF{c}\text{ and }\BF{G}\bm{\alpha}=h
\end{align}

\QEDB

\horrule{0.5pt}

\subsection*{Problem 4}

I wrote a \url{Q04.py} to help me get the answer. By using python package \url{sympy} and
\begin{align}
\BF{w}&=\sum_{n=1}^{N}\alpha_ny_nK\ParTh{\BF{x}_n,\Matrix{c}{x_1\\x_2}}+b\\
b&=y_s-\sum_{n=1}^{N}\alpha_ny_nK\ParTh{\BF{x}_n,\BF{x}_s}
\end{align}
we have
\begin{align}
\BF{w}=\dfrac{1}{9}\ParTh{8x^2_1 - 16x_1 + 6x^2_2-15}
\end{align}

\QEDB

\horrule{0.5pt}

\subsection*{Problem 5}

Since kernel function $K\ParTh{\BF{x},\BF{x}^\prime}=\ParTh{1+\BF{x}^T\BF{x}^\prime}^2$ is different from $\BF{z}=\ParTh{\phi\ParTh{\BF{x}},\phi\ParTh{\BF{x}}}$, the curves should be different in the $\mathcal{X}$ space.

\QEDB

\horrule{0.5pt}

\subsection*{Problem 6}

Since $\VecAbsVal{\BF{x}_n-\BF{c}}^2\leq R^2$, $\forall n$, the constraint to maximize is
\begin{align}
\VecAbsVal{\BF{x}_n-\BF{c}}^2-R^2\leq0
\end{align}
so $L\ParTh{R,\BF{c},\bm{\lambda}}$ is
\begin{align}
L\ParTh{R,\BF{c},\bm{\lambda}}=R^2+\sum_{n=1}^{N}\lambda_n\ParTh{\VecAbsVal{\BF{x}_n-\BF{c}}^2-R^2}
\end{align}

\QEDB

\horrule{0.5pt}

\subsection*{Problem 7}

At the optimal $(R,\BF{c},\bm{\lambda})$,
\begin{align}
\dfrac{\partial L}{\partial R}&=2R-2R\sum_{n=1}^{N}\lambda_n=0\Rightarrow\sum_{n=1}^{N}\lambda_n=1\text{ or }R=0\\
\label{KKTC}
\dfrac{\partial L}{\partial\BF{c}}&=2\sum_{n=1}^{N}\lambda_{n}\ParTh{\BF{c}-\BF{x}_{n}}=\BF{0}\Rightarrow\BF{c}=\Divide{\ParTh{\sum_{n=1}^{N}\lambda_n\BF{x}_n}}{\ParTh{\sum_{n=1}^{N}\lambda_n}}\text{ if }\sum_{n=1}^{N}\lambda_n\neq0
%\Rightarrow\lambda_n=0\text{ or }\BF{x}_n=\BF{c}
\end{align}
So the KKT conditions are
\begin{enumerate}
	\item primal feasible: $\VecAbsVal{\BF{x}_n-\BF{c}}^2\leq R^2$.
	\item dual feasible: $\lambda_n\geq0$.
	\item dual-inner optimal: if $R\neq0$, $\sum_{n=1}^{N}\lambda_n=1$ and $\BF{c}=\sum_{n=1}^{N}\lambda_n\BF{x}_n$.
	\item primal-inner optimal: $\lambda_n\ParTh{\VecAbsVal{\BF{x}_n-\BF{c}}^2-R^2}=0$.
\end{enumerate}

\QEDB

\horrule{0.5pt}

\subsection*{Problem 8}

From Problem 6, we have
\begin{align}
L\ParTh{R,\BF{c},\bm{\lambda}}&=R^2+\sum_{n=1}^{N}\lambda_n\ParTh{\VecAbsVal{\BF{x}_n-\BF{c}}^2-R^2}=R^2+\sum_{n=1}^{N}\lambda_n\VecAbsVal{\BF{x}_n-\BF{c}}^2-\sum_{n=1}^{N}\lambda_nR^2\\
&=R^2-R^2+\sum_{n=1}^{N}\lambda_n\VecAbsVal{\BF{x}_n-\BF{c}}^2=\sum_{n=1}^{N}\lambda_n\VecAbsVal{\BF{x}_n-\BF{c}}^2
\end{align}
where $\sum_{n=1}^{N}\lambda_n=1$ since $R\neq0$.

Also, from (\ref{KKTC}), we have $\BF{c}=\sum_{n=1}^{N}\lambda_n\BF{x}_n$. Hence
\begin{align}
\text{Objective}\ParTh{\lambda}=\sum_{n=1}^{N}\lambda_n\VecAbsVal{\BF{x}_n-\sum_{m=1}^{N}\lambda_m\BF{x}_m}^2
\end{align}

\QEDB

\horrule{0.5pt}

\subsection*{Problem 9}

We have
\begin{align}
\sum_{n=1}^{N}\lambda_n\VecAbsVal{\BF{x}_n-\BF{c}}^2&=\sum_{n=1}^{N}\lambda_n\ParTh{\BF{x}^T_n\BF{x}_n-\BF{x}^T_n\BF{c}-\BF{c}^T\BF{x}_n+\BF{c}^T\BF{c}}\\
&=\sum_{n=1}^{N}\lambda_n\ParTh{\BF{x}^T_n\BF{x}_n-\BF{x}^T_n\sum_{m=1}^{N}\lambda_m\BF{x}_m-\ParTh{\sum_{m=1}^{N}\lambda_m\BF{x}_m}^T\BF{x}_n+\VecAbsVal{\sum_{m=1}^{N}\lambda_m\BF{x}_m}^2}
\end{align}
So
\begin{align}
&\sum_{n=1}^{N}\lambda_n\VecAbsVal{\phi\ParTh{\BF{x}_n}-\phi\ParTh{\BF{c}}}^2\\
=&\sum_{n=1}^{N}\lambda_nK\ParTh{\BF{x}_n,\BF{x}_n}-2\sum_{n=1}^{N}\sum_{m=1}^{N}\lambda_n\lambda_mK\ParTh{\BF{x}_n,\BF{x}_m}+\sum_{n=1}^{N}\sum_{m=1}^{N}\lambda_n\lambda_mK\ParTh{\BF{x}_n,\BF{x}_m}\\
=&\sum_{n=1}^{N}\lambda_nK\ParTh{\BF{x}_n,\BF{x}_n}-\sum_{n=1}^{N}\sum_{m=1}^{N}\lambda_n\lambda_mK\ParTh{\BF{x}_n,\BF{x}_m}
\end{align}

\QEDB

\horrule{0.5pt}

\subsection*{Problem 10}

From primal-inner optimal condition, pick some $\lambda_i>0$, we have
\begin{align}
\VecAbsVal{\BF{x}_i-\BF{c}}^2=R^2
\end{align}
so
\begin{align}
R^2&=\BF{x}^T_i\BF{x}_i-\BF{x}^T_i\sum_{m=1}^{N}\lambda_m\BF{x}_m-\ParTh{\sum_{m=1}^{N}\lambda_m\BF{x}_m}^T\BF{x}_i+\sum_{n=1}^{N}\sum_{m=1}^{N}\lambda_n\lambda_m\BF{x}^T_n\BF{x}_m\\
&=K\ParTh{\BF{x}_i,\BF{x}_i}-2\sum_{m=1}^{N}\lambda_mK\ParTh{\BF{x}_i,\BF{x}_m}+\sum_{n=1}^{N}\sum_{m=1}^{N}\lambda_n\lambda_mK\ParTh{\BF{x}_n,\BF{x}_m}\\
\Rightarrow R&=\sqrt{K\ParTh{\BF{x}_i,\BF{x}_i}-2\sum_{m=1}^{N}\lambda_mK\ParTh{\BF{x}_i,\BF{x}_m}+\sum_{n=1}^{N}\sum_{m=1}^{N}\lambda_n\lambda_mK\ParTh{\BF{x}_n,\BF{x}_m}}
\end{align}
where $R>0$.

\QEDB

\horrule{0.5pt}

\subsection*{Problem 11}

\underline{Claim}: Let $\tilde{\BF{w}}=\Matrix{c}{\BF{w}\\\sqrt{2C}\cdot\bm{\xi}}$ and $\tilde{\BF{x}}_n=\Matrix{c}{\BF{x}_n\\v_1\\v_2\\\vdots\\v_N}$, where $v_i=\dfrac{1}{\sqrt{2C}}\BooBrackets{i=n}$.

\underline{Proof of Claim}:

First, we have
\begin{align}
\dfrac{1}{2}\tilde{\BF{w}}^T\tilde{\BF{w}}&=\dfrac{1}{2}\Matrix{cc}{\BF{w}^T&\sqrt{2C}\cdot\bm{\xi}^T}\Matrix{c}{\BF{w}\\\sqrt{2C}\cdot\bm{\xi}}=\dfrac{1}{2}\BF{w}^T\BF{w}+C\bm{\xi}^T\bm{\xi}=\dfrac{1}{2}\BF{w}^T\BF{w}+C\sum_{n=1}^{N}\xi^2_n
\end{align}
And $\ParTh{P_2}$ can be rewritten as
\begin{align}
\min_{\tilde{\BF{w}},b,\bm{\xi}}\ParTh{\dfrac{1}{2}\tilde{\BF{w}}^T\tilde{\BF{w}}}
\end{align}
Then we have
\begin{align}
\tilde{\BF{w}}^T\tilde{\BF{x}}_n&=\Matrix{cc}{\BF{w}^T&\sqrt{2C}\cdot\bm{\xi}^T}\Matrix{c}{\BF{x}_n\\v_1\\v_2\\\vdots\\v_N}=\Matrix{cc}{\BF{w}^T&\sqrt{2C}\cdot\bm{\xi}^T}\Matrix{c}{\BF{x}_n\\0\\\vdots\\\underbrace{\dfrac{1}{\sqrt{2C}}}_{i=n}\\\vdots\\0}\\
&=\BF{w}^T\BF{x}_n+0+\cdots+\xi_n+\cdots+0=\BF{w}^T\BF{x}_n+\xi_n
\end{align}
So
\begin{align}
&y_n\ParTh{\BF{w}^T\BF{x}_n+b}\geq1-\xi_n\\
\Rightarrow ~&y_n\ParTh{\BF{w}^T\BF{x}_n+b}+\xi_n=y_n\ParTh{\BF{w}^T\BF{x}_n+\xi_n+b}=y_n\ParTh{\tilde{\BF{w}}^T\tilde{\BF{x}}_n+b}\geq1
\end{align}
where $y_n\xi_n=\xi_n$ is due to $y_n\in\CBrackets{+1,-1}$.

\QEDB

\horrule{0.5pt}

\subsection*{Problem 12}

\QEDB

\horrule{0.5pt}

\subsection*{Problem 13}

\QEDB

\horrule{0.5pt}

\subsection*{Problem 14}

\QEDB

\horrule{0.5pt}

\subsection*{Problem 15}

\QEDB

\horrule{0.5pt}

\subsection*{Problem 16}

\QEDB

\horrule{0.5pt}

\subsection*{Problem 17}

\QEDB

\horrule{0.5pt}

\subsection*{Problem 18}

\QEDB

\horrule{0.5pt}

\subsection*{Problem 19}

\QEDB

\horrule{0.5pt}

\subsection*{Problem 20}

\QEDB

\horrule{0.5pt}

\section*{Reference}

\begin{enumerate}

\item[{[1]}] Lecture Notes by Hsuan-Tien LIN, Department of Computer Science and Information Engineering, National Taiwan University, Taipei 106, Taiwan.

\item[{[2]}] Quadratic Programming with Python and CVXOPT

\url{https://courses.csail.mit.edu/6.867/wiki/images/a/a7/Qp-cvxopt.pdf}

\end{enumerate}

\end{document}