\documentclass[12pt]{article}

% Use packages %
\usepackage{graphicx, courier, amsmath, amssymb, amscd, amsfonts, mathtools, bm, esint, leftidx, extarrows, latexsym, relsize, color, tikz, comment, stmaryrd}
\usepackage[obeyspaces]{url}% http://ctan.org/pkg/url

% Set length %
\setlength{\textwidth}{160mm}
\setlength{\textheight}{235mm}
\setlength{\oddsidemargin}{-0mm}
\setlength{\topmargin}{-10mm}

% Define h-bar %
\newsavebox{\myhbar}
\savebox{\myhbar}{$\hbar$}
\renewcommand*{\hbar}{\mathalpha{\usebox{\myhbar}}}

% Chinese input %
%\usepackage{xeCJK} 
%\setCJKmainfont{微軟正黑體}
%\usepackage[T1]{fontenc}
%\makeatletter

% Equation number %
%\@addtoreset{equation}{section} 
%\renewcommand\theequation{{\thesection}.{\arabic{equation}}}
%\makeatletter 

% Helper Command %
\newcommand{\argmin}{\operatornamewithlimits{argmin}}
\newcommand{\rmnum}[1]{\romannumeral #1} 
\newcommand{\Rmnum}[1]{\expandafter\@slowromancap\romannumeral #1@}
\newcommand{\overbar}[1]{\mkern 1.5mu\overline{\mkern-1.5mu#1\mkern-1.5mu}\mkern 1.5mu}
\makeatother
\newcommand*{\QEDA}{\hfill\ensuremath{\blacksquare}}
\newcommand*{\QEDB}{\hfill\ensuremath{\square}}
\newcommand*{\BmVert}{\bigm\vert}
\newcommand{\bigslant}[2]{{\raisebox{.2em}{$#1$}\left/\raisebox{-.2em}{$#2$}\right.}}
\newcommand{\Nelements}[3]{\left\{ #1, ~ #2, \ldots, ~ #3 \right\}}
\newcommand{\CBrackets}[1]{\left\{#1\right\}}
\newcommand{\SBrackets}[1]{\left[#1\right]}
\newcommand{\ParTh}[1]{\left(#1\right)}
\newcommand{\Ceil}[1]{\left\lceil#1\right\rceil}
\newcommand{\Floor}[1]{\left\lfloor#1\right\rfloor}
\newcommand{\BF}[1]{{\bf#1}}
\newcommand{\Inverse}[1]{{#1}^{-1}}
\newcommand{\Generator}[1]{\left\langle#1\right\rangle}
\newcommand{\AbsVal}[1]{\left|#1\right|}
\newcommand{\VecAbsVal}[1]{\left\|#1\right\|}
\newcommand{\BSlash}[2]{\left.#1\middle\backslash#2\right.}
\newcommand{\Divide}[2]{\left.#1\middle/#2\right.}
\newcommand{\SciNum}[2]{#1\times{10}^{#2}}
\newcommand{\Matrix}[2]{\ParTh{\begin{array}{#1}#2\end{array}}}
\newcommand{\MatrixTwo}[4]{\ParTh{\begin{array}{cc}{#1}&{#2}\\{#3}&{#4}\end{array}}}
\newcommand{\MatrixNByN}[1]{\Matrix{cccc}{{#1}_{11} & {#1}_{12} & \cdots & {#1}_{1n} \\ {#1}_{21} & {#1}_{22} & \cdots & {#1}_{2n} \\ \vdots & \vdots & \ddots & \vdots \\ {#1}_{n1} & {#1}_{n2} & \cdots & {#1}_{nn}}}
\newcommand{\ndiv}{\hspace{-4pt}\not|\hspace{2pt}}
\newcommand{\eqdef}{\xlongequal{\text{def}}}%
\newcount\arrowcount
\newcommand\arrows[1]{\global\arrowcount#1 \ifnum\arrowcount>0
\begin{matrix}\expandafter\nextarrow\fi}
\newcommand\nextarrow[1]{\global\advance\arrowcount-1 \ifx\relax#1\relax\else \xrightarrow{#1}\fi\ifnum\arrowcount=0 \end{matrix}\else\\\expandafter\nextarrow\fi}
\newcommand{\horrule}[1]{\rule{\linewidth}{#1}}

% Tikz settings %
\usetikzlibrary{shapes,arrows}
\tikzstyle{decision} = [diamond, draw, fill=white!20, text width=4.5em, text badly centered, node distance=3cm, inner sep=0pt]
\tikzstyle{block}    = [rectangle, draw, fill=white!20, text width=8em, text centered, rounded corners, minimum height=4em]
\tikzstyle{point}    = [fill = white!20, minimum size=0.5cm]
\tikzstyle{line}     = [draw, -latex']
\tikzstyle{mapsto}   = [draw, |->]
\tikzstyle{cloud}    = [draw, ellipse,fill=red!20, node distance=3cm, minimum height=2em]

\begin{document}

\baselineskip 6.5mm
\setlength{\parindent}{0pt}
\title{ 
\normalfont \normalsize 
\horrule{0.5pt} \\[0.4cm]
\huge { \Huge Machine Learning \\ \large Answer Sheet for Homework 3 }\\ % The assignment title
\horrule{2pt} \\ [0.5cm]
}
\author{ { \Large Da-Min HUANG } \\
{\small R04942045} \\
{\small\textit{Graduate Institute of Communication Engineering, National Taiwan University}}
}
%\date{November 2, 2015}
%\allowdisplaybreaks[4]
\maketitle

\subsection*{Problem 1}

Set $\sigma=0.1$ and $d=8$, then we can rewrite the formula to be
\begin{align}
\mathbb{E}_{\mathcal{D}}\SBrackets{E_{\text{in}}\ParTh{\BF{w}_{\text{lin}}}}=0.01\ParTh{1-\dfrac{9}{N}}>0.008\Rightarrow0.2>\dfrac{9}{N}\Rightarrow N>45
\end{align}

\QEDB

\horrule{0.5pt}

\subsection*{Problem 2}

\begin{enumerate}
	\item[(a)]
	\begin{comment}
	Consider the physical meaning of $\BF{H}$. The projection won't cause the result to be less than zero (one point). So $\BF{H}$ is positive semi-definite.
	\end{comment}
	$\BF{H}$ is positive semi-definite $\Leftrightarrow$ All eigenvalues is non-negative. Refer to choice (c), we have shown the properties.
	\item[(b)]
	\begin{comment}
	Consider the physical meaning of $\BF{H}$. The inverse of projection is not injective, which implies that $\BF{H}$ is not always invertible.
	\end{comment}
	Consider $\BF{X}=\Matrix{ccc}{1&2&1\\0&1&1\\1&1&1}$, we have
	\begin{align}
	\BF{H}=\BF{X}\ParTh{\BF{X}^T\BF{X}}^{-1}\BF{X}^T=\Matrix{ccc}{0&0&1\\0&0&1\\1&1&-1}
	\end{align}
	which has no inverse.
	
	Also, consider the physical meaning of $\BF{H}$. The inverse of projection is not injective, which implies that $\BF{H}$ is not always invertible.
	\item[(c)] Refer to choice (e), we have $\BF{H}^2=\BF{H}$. Suppose $\lambda$ is the eigenvalue of some non-zero vector $\vec{v}$,
	\begin{align}
	\BF{H}\vec{v}=\lambda\vec{v}=\BF{H}^2\vec{v}=\BF{H}\ParTh{\lambda\vec{v}}=\lambda\ParTh{\BF{H}\vec{v}}=\lambda^2\vec{v}\Rightarrow\lambda^2=\lambda
	\end{align}
	Hence, the possible results of $\lambda$ is 1 or 0.
	\item[(d)] Consider the physical meaning of $\BF{H}$. Since there are $d$ features, so at least $d+1$ (since the feature vector contains $x_0$ term) eigenvalues are 1.
	\item[(e)] By the definition of $\BF{H}$, we have
	\begin{align}
	\BF{H}^2&=\ParTh{\BF{X}\ParTh{\BF{X}^T\BF{X}}^{-1}\BF{X}^T}\ParTh{\BF{X}\ParTh{\BF{X}^T\BF{X}}^{-1}\BF{X}^T}\\
	&=\BF{X}\ParTh{\BF{X}^T\BF{X}}^{-1}\ParTh{\ParTh{\BF{X}^T\BF{X}}\ParTh{\BF{X}^T\BF{X}}^{-1}}\BF{X}^T\\
	&=\BF{X}\ParTh{\BF{X}^T\BF{X}}^{-1}\BF{X}^T=\BF{H}
	\end{align}
	So
	\begin{align}
	\BF{H}^2=\BF{H}\Rightarrow \BF{H}^{1126}=\BF{H}
	\end{align}
\end{enumerate}

\QEDB

\horrule{0.5pt}

\subsection*{Problem 3}

If $\text{sign}\ParTh{\BF{w}^T\BF{x}}\neq y$, then $y\BF{w}^T\BF{x}<0$ since the sign of $y$ and $\BF{w}^T\BF{x}$ are different. Similarly, if $\text{sign}\ParTh{\BF{w}^T\BF{x}}=y$, then $y\BF{w}^T\BF{x}\geq0$.

\underline{Claim}: $\ParTh{\max\ParTh{0,1-y\BF{w}^T\BF{x}}}^2$ is an upper bound.

\underline{Proof of claim}:

Consider the following cases.
\begin{enumerate}
	\item $\SBrackets{\text{sign}\ParTh{\BF{w}^T\BF{x}}\neq y}=0$
	
	Then $y\BF{w}^T\BF{x}\geq0$. Hence $\ParTh{\max\ParTh{0,1-y\BF{w}^T\BF{x}}}^2\geq0$, which bounds $\SBrackets{\text{sign}\ParTh{\BF{w}^T\BF{x}}\neq y}$.
	\item $\SBrackets{\text{sign}\ParTh{\BF{w}^T\BF{x}}\neq y}=1$
	
	Then $y\BF{w}^T\BF{x}<0$. Hence $\ParTh{\max\ParTh{0,1-y\BF{w}^T\BF{x}}}^2=\ParTh{1-y\BF{w}^T\BF{x}}^2>1$, which bounds $\SBrackets{\text{sign}\ParTh{\BF{w}^T\BF{x}}\neq y}$.
\end{enumerate}

\QEDB

\horrule{0.5pt}

\subsection*{Problem 4}

Set $y\BF{w}^T\BF{x}\coloneqq z$. Consider $\max\ParTh{0,-y\BF{w}^T\BF{x}}=\max\ParTh{0,-z}\coloneqq f\ParTh{z}$. We have $f\ParTh{z}=-z$ if $z\leq0$, else $f\ParTh{z}=0$. So 
\begin{align}
\lim\limits_{z\rightarrow0^{-}}\dfrac{f\ParTh{z}-f\ParTh{0}}{z-0}=\dfrac{-z-0}{z-0}=-1,~\lim\limits_{z\rightarrow0^{+}}\dfrac{f\ParTh{z}-f\ParTh{0}}{z-0}=\dfrac{0-0}{z-0}=0
\end{align}
Hence, $f\ParTh{z}$ is not differentiable at $z=0$.

\QEDB

\horrule{0.5pt}

\subsection*{Problem 5}

\underline{Calim}: $\max\ParTh{0, -y\BF{w}^T\BF{x}}$ results in PLA.

\underline{Proof of claim}:

Consider following cases.
\begin{enumerate}
	\item $y = \text{sign}\ParTh{\BF{w}^T\BF{x}}$.
	Then we have $y\BF{w}^T\BF{x} > 0$. So
	\begin{align}
	\max\ParTh{0, -y\BF{w}^T\BF{x}}=0
	\end{align}
	\item $y \neq \text{sign}\ParTh{\BF{w}^T\BF{x}}$.
	Then we have $y\BF{w}^T\BF{x} < 0$. So
	\begin{align}
	\max\ParTh{0, -y\BF{w}^T\BF{x}}=-y\BF{w}^T\BF{x}\Rightarrow-\nabla_{\BF{w}}\max\ParTh{0, -y\BF{w}^T\BF{x}}=y\BF{x}
	\end{align}
\end{enumerate}
\begin{comment}
Set $\eta=1$ and let $\BF{w}^T_t\BF{x}_n$ be large enough, we have
\begin{align}
\lim\limits_{\BF{w}^T_t\BF{x}_n\rightarrow\infty}\theta\ParTh{-y_n\BF{w}^T_t\BF{x}_n}&=\lim\limits_{\BF{w}^T_t\BF{x}_n\rightarrow\infty}\dfrac{1}{1+\exp\ParTh{y_n\BF{w}^T_t\BF{x}_n}}=\left\{\begin{array}{ll}
1,&\text{if sign}\ParTh{\BF{w}^T_t\BF{x}_n}\neq y_n\\
0,&\text{if sign}\ParTh{\BF{w}^T_t\BF{x}_n}= y_n
\end{array}\right.\\
&=\left\llbracket{y_n\neq\text{sign}\ParTh{\BF{w}^T_t\BF{x}_n}}\right\rrbracket
\end{align}
\end{comment}

\QEDB

\horrule{0.5pt}

\subsection*{Problem 6}

\begin{align}
\nabla E(0,0) &= \left.\ParTh{\frac{\partial E}{\partial u},\frac{\partial E}{\partial v}}\right|_{\ParTh{0,0}}\\
&= \left.\ParTh{e^{u}+ve^{uv}+2u-2v-3,2e^{2v}+ue^{uv}-2u+4v-2}\right|_{\ParTh{0,0}}\\
&=\ParTh{-2,0}
\end{align}

\QEDB

\horrule{0.5pt}

\subsection*{Problem 7}

\begin{align}
\ParTh{u_1,v_1}&=\ParTh{0,0}-0.01\nabla E\ParTh{0,0}=\ParTh{0.02,0}\\
\ParTh{u_2,v_2}&=\ParTh{0.02,0}-0.01\nabla E\ParTh{0.02,0}\approx\ParTh{0.039398,0.0002}\\
\ParTh{u_3,v_3}&\approx\ParTh{0.039398,0.0002}-0.01\nabla E\ParTh{0.039398,0.0002}\\&\approx\ParTh{0.0582102,0.000577975}\\
\ParTh{u_4,v_4}&\approx\ParTh{0.0764524,0.00111381}\\
\ParTh{u_5,v_5}&\approx\ParTh{0.09414,0.00178911}\\
E\ParTh{u_5,v_5}&\approx2.825
\end{align}

\QEDB

\horrule{0.5pt}

\subsection*{Problem 8}

\begin{align}
\nabla E(0,0) &= \ParTh{\frac{\partial E}{\partial u},\frac{\partial E}{\partial v}}\\
&=\ParTh{e^{u}+ve^{uv}+2u-2v-3,2e^{2v}+ue^{uv}-2u+4v-2}
\end{align}
From this we compute the Hessian matrix
\begin{align}
\nabla^2E\ParTh{u,v}=\Matrix{cc}{e^{u}+v^2e^{uv}+2&\ParTh{uv+1}e^{uv}-2\\\ParTh{uv+1}e^{uv}-2&4e^{2v}+u^2e^{uv}+4}
\end{align}
So
\begin{align}
\hat{E}\ParTh{\Delta u,\Delta v}&=E\ParTh{0,0}+\nabla E\ParTh{0,0}\cdot\ParTh{\Delta u,\Delta v}+\dfrac{1}{2}\ParTh{\Delta u,\Delta v}\nabla^2E\ParTh{0,0}\Matrix{c}{\Delta u\\\Delta v}\\
&=3-2\Delta u+\dfrac{1}{2}\ParTh{\Delta u,\Delta v}\Matrix{cc}{3&-1\\-1&8}\Matrix{c}{\Delta u\\\Delta v}\\
&=\dfrac{3}{2}\ParTh{\Delta u}^2+4\ParTh{\Delta v}^2-\Delta u\Delta v-2\Delta u+0\Delta v+3
\end{align}

\QEDB

\horrule{0.5pt}

\subsection*{Problem 9}

\underline{Claim}: $-\ParTh{\nabla^2E\ParTh{u,v}}^{-1}\nabla E\ParTh{u,v}$ is the Newton direction.

\underline{Proof of claim}:
\begin{align}
\dfrac{\partial\hat{E}\ParTh{\Delta u,\Delta v}}{\partial\ParTh{\Delta u,\Delta v}}&=\nabla E\ParTh{u,v}+\nabla^2E\ParTh{u,v}\ParTh{\Delta u,\Delta v}=0\\
\Rightarrow\ParTh{\Delta u,\Delta v}&=-\ParTh{\nabla^2E\ParTh{u,v}}^{-1}\nabla E\ParTh{u,v}
\end{align}
\begin{comment}
Now we have
\begin{align}
\hat{E}\ParTh{\Delta u,\Delta v}&=\dfrac{3}{2}\ParTh{\Delta u}^2+4\ParTh{\Delta v}^2-\Delta u\Delta v-2\Delta u+3
\end{align}
So
\begin{align}
\left.\begin{array}{l}
\dfrac{\partial\hat{E}}{\partial\ParTh{\Delta u}}=3\Delta u-\Delta v-2=0\\
\dfrac{\partial\hat{E}}{\partial\ParTh{\Delta v}}=8\Delta v-\Delta u=0\\
\end{array}\right\}\Rightarrow\Delta u=\dfrac{16}{23},~\Delta v=\dfrac{2}{23}
\end{align}
\end{comment}

\QEDB

\horrule{0.5pt}

\subsection*{Problem 10}

\begin{align}
\ParTh{u_1,v_1}&\approx\ParTh{0.695652173913, 0.0869565217391}\\
\ParTh{u_2,v_2}&\approx\ParTh{0.613762221112, 0.0711078990173}\\
\ParTh{u_3,v_3}&\approx\ParTh{0.611812859879, 0.0705000613365}\\
\ParTh{u_4,v_4}&\approx\ParTh{0.611811717261, 0.0704995471019}\\
\ParTh{u_5,v_5}&\approx\ParTh{0.61181171726, 0.0704995471016}\\
E\ParTh{u_5,v_5}&\approx2.36082334564
\end{align}
This equals to the value of Problem 7 after 746 updates.

\QEDB

\horrule{0.5pt}

\subsection*{Problem 11}

Write a program Q11.py to test, $\CBrackets{\BF{x}_1,\BF{x}_2,\BF{x}_3,\BF{x}_4,\BF{x}_5,\BF{x}_6}$ is the biggest subset that can be shattered by the union of quadratic, linear, or constant hypotheses of $\BF{x}$.
\begin{comment}
If $\BF{x}_1$ and $\BF{x}_3$ are on the same side, then $\BF{x}_2$ and $\BF{x}_4$ can't be the same side. So $\CBrackets{\BF{x}_1,\BF{x}_2,\BF{x}_3}$ is the biggest subset that can be shattered by the union of quadratic, linear, or constant hypotheses of $\BF{x}$.
\begin{align}
\BF{x}_4&=-\BF{x}_2\\
\BF{x}_5&=\BF{x}_1+\BF{x}_3\\
\BF{x}_6&=\dfrac{1}{2}\ParTh{\BF{x}_1+\BF{x}_2}
\end{align}
So $\CBrackets{\BF{x}_1,\BF{x}_2,\BF{x}_3}$ is the biggest subset that can be shattered by the union of quadratic, linear, or constant hypotheses of $\BF{x}$.
\end{comment}

\QEDB

\horrule{0.5pt}

\subsection*{Problem 12}

By the transform, we have $\ParTh{\Phi\ParTh{\BF{x}}}_i=z_i=\ParTh{0,\ldots,\underbrace{1}_{i\text{-th term}},\ldots,0}$. To shatter the original $N$ points, we can assign $w_i$ to be positive or negative to get $\BF{x}_i$ $\circ$ or $\times$.

So, this transform shatter any $N$ points. Hence $d_{vc}\ParTh{\mathcal{H}_\Phi}=\infty$.

\QEDB

\horrule{0.5pt}

\subsection*{Problem 13}

The average $E_{\text{in}}$ is 0.503979.

\QEDB

\horrule{0.5pt}

\subsection*{Problem 14}

The returned $\BF{w}_{\text{Lin}}=\ParTh{-1.00134023,  0.07529962  0.01237623,  0.0812999,   1.69273348,  1.53664765}$.

\QEDB

\horrule{0.5pt}

\subsection*{Problem 15}

The average $E_{\text{out}}=0.127198$.

\QEDB

\horrule{0.5pt}

\subsection*{Problem 16}

Sum the minimized negative log likelihood $h_y$, which is $\min_y\ParTh{-\ln\ParTh{h_y}}$, we have
\begin{align}
E_{\text{in}}&=\dfrac{1}{N}\sum_{n=1}^{N}\ParTh{-\ln\ParTh{\dfrac{\exp\ParTh{\BF{w}^T_{y_n}\BF{x}_n}}{\sum_{i=1}^{K}\exp\ParTh{\BF{w}^T_i\BF{x}_n}}}}\\
&=\dfrac{1}{N}\sum_{n=1}^{N}\ParTh{\ln\ParTh{\sum_{i=1}^{K}\exp\ParTh{\BF{w}^T_i\BF{x}_n}}-\BF{w}^T_{y_n}\BF{x}_n}
\end{align}

\QEDB

\horrule{0.5pt}

\subsection*{Problem 17}

\begin{align}
\dfrac{\partial E_{\text{in}}}{\partial \BF{w}_i}&=\dfrac{1}{N}\sum_{n=1}^{N}\dfrac{\partial}{\partial \BF{w}_i}\ParTh{\ln\ParTh{\sum_{i=1}^{K}\exp\ParTh{\BF{w}^T_i\BF{x}_n}}-\BF{w}^T_{y_n}\BF{x}_n}\\
&=\dfrac{1}{N}\sum_{n=1}^{N}\ParTh{\dfrac{1}{\sum_{i=1}^{K}\exp\ParTh{\BF{w}^T_i\BF{x}_n}}\dfrac{\partial}{\partial \BF{w}_i}\exp\ParTh{\BF{w}^T_i\BF{x}_n}-\dfrac{\partial}{\partial \BF{w}_i}\BF{w}^T_{y_n}\BF{x}_n}\\
&=\dfrac{1}{N}\sum_{n=1}^{N}\ParTh{\dfrac{\exp\ParTh{\BF{w}^T_i\BF{x}_n}}{\sum_{i=1}^{K}\exp\ParTh{\BF{w}^T_i\BF{x}_n}}\BF{x}_n-\left\llbracket y_n=i\right\rrbracket\BF{x}_n}\\
&=\dfrac{1}{N}\sum_{n=1}^{N}\ParTh{h_i\ParTh{\BF{x}_n}-\left\llbracket y_n=i\right\rrbracket}\BF{x}_n
\end{align}

\QEDB

\horrule{0.5pt}

\subsection*{Problem 18}

The $E_{\text{out}}=0.475$.

\QEDB

\horrule{0.5pt}

\subsection*{Problem 19}

The $E_{\text{out}}=0.220$.

\QEDB

\horrule{0.5pt}

\subsection*{Problem 20}

The $E_{\text{out}}=0.473$.

\QEDB

\horrule{0.5pt}

\section*{Reference}

\begin{enumerate}

\item[{[1]}] Lecture Notes by Hsuan-Tien LIN, Department of Computer Science and Information Engineering, National Taiwan University, Taipei 106, Taiwan.

%\item[{[2]}] Three proofs of Sauer-Shelah Lemma. (n. d. ). Retrieved Fall, 2010, from \url{http://www.cse.buffalo.edu/~hungngo/classes/2010/711/lectures/sauer.pdf}

\end{enumerate}

\end{document}